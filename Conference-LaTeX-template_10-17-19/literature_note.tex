\documentclass[conference]{IEEEtran}
\IEEEoverridecommandlockouts
% The preceding line is only needed to identify funding in the first footnote. If that is unneeded, please comment it out.
\usepackage{cite}
\usepackage{amsmath,amssymb,amsfonts}
\usepackage{algorithmic}
\usepackage{graphicx}
\usepackage{textcomp}
\usepackage{xcolor}
\def\BibTeX{{\rm B\kern-.05em{\sc i\kern-.025em b}\kern-.08em
    T\kern-.1667em\lower.7ex\hbox{E}\kern-.125emX}}
\begin{document}

\title{The Literature of Path Planning*\\
{\footnotesize \textsuperscript{*}Note: Sub-titles are not captured in Xplore and
should not be used}
\thanks{Identify applicable funding agency here. If none, delete this.}
}

\author{\IEEEauthorblockN{1\textsuperscript{st} Wenxing Lan}
\IEEEauthorblockA{\textit{dept. name of organization (of Aff.)} \\
\textit{name of organization (of Aff.)}\\
City, Country \\
email address or ORCID}
\and
\IEEEauthorblockN{2\textsuperscript{nd} Given Name Surname}
\IEEEauthorblockA{\textit{dept. name of organization (of Aff.)} \\
\textit{name of organization (of Aff.)}\\
City, Country \\
email address or ORCID}
\and
\IEEEauthorblockN{3\textsuperscript{rd} Given Name Surname}
\IEEEauthorblockA{\textit{dept. name of organization (of Aff.)} \\
\textit{name of organization (of Aff.)}\\
City, Country \\
email address or ORCID}

}

\maketitle

\begin{abstract}
This document is a model and instructions for \LaTeX.
This and the IEEEtran.cls file define the components of your paper [title, text, heads, etc.]. *CRITICAL: Do Not Use Symbols, Special Characters, Footnotes, 
or Math in Paper Title or Abstract.
\end{abstract}

\begin{IEEEkeywords}
component, formatting, style, styling, insert
\end{IEEEkeywords}

\section{Local path planning in a complex environment for self-driving car}
\subsection{Abstract}
\begin{enumerate}
	\item This paper introduces an local path planning algorithm for the self-driving car in a complex environment.
	\item Novelty:
		\begin{enumerate}
			\item The novel path representation.
			\item The collision detection and the path modification using a voronoi cell.
			\item The novel path representation provides convenience for checking the collision and modifying the path and continuous control input for steering wheel rather than way point navigation.
		\end{enumerate}
\end{enumerate}
\subsection{Introduction}
\begin{enumerate}
	\item Path planning can be categorized into Potential-field approach\cite{realtime}, Roadmap based approach, and Cell decomposition based approach.
		\begin{enumerate}
			\item The cell decomposition based approach is to divide free space where no collision against obstacles is found into a certain size of cells and find a path by connecting adjacent cells.
			\item The roadmap based approach is divided into deterministic roadmap and probabilistic roadmap. A typical example of the probabilistic roadmap is using RRT.
		\end{enumerate}
	\item 
\end{enumerate}


\section*{Acknowledgment}

The preferred spelling of the word ``acknowledgment'' in America is without 
an ``e'' after the ``g''. Avoid the stilted expression ``one of us (R. B. 
G.) thanks $\ldots$''. Instead, try ``R. B. G. thanks$\ldots$''. Put sponsor 
acknowledgments in the unnumbered footnote on the first page.


\bibliographystyle{IEEEtran}
\bibliography{literature} 

\end{document}

