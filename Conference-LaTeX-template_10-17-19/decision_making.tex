\documentclass[conference]{IEEEtran}
\IEEEoverridecommandlockouts
% The preceding line is only needed to identify funding in the first footnote. If that is unneeded, please comment it out.
\usepackage{cite}
\usepackage{amsmath,amssymb,amsfonts}
\usepackage{algorithmic}
\usepackage{graphicx}
\usepackage{textcomp}
\usepackage{xcolor}
\usepackage{url}
\def\BibTeX{{\rm B\kern-.05em{\sc i\kern-.025em b}\kern-.08em
    T\kern-.1667em\lower.7ex\hbox{E}\kern-.125emX}}
\begin{document}

\title{The Decision Making of Self-driving Vehicles: A Survey}

\author{\IEEEauthorblockN{1\textsuperscript{st} Wenxing Lan}
\IEEEauthorblockA{\textit{dept. name of organization (of Aff.)} \\
\textit{name of organization (of Aff.)}\\
City, Country \\
email address or ORCID}


}

\maketitle

\begin{abstract}
This document is a model and instructions for \LaTeX.
This and the IEEEtran.cls file define the components of your paper [title, text, heads, etc.]. *CRITICAL: Do Not Use Symbols, Special Characters, Footnotes, 
or Math in Paper Title or Abstract.
\end{abstract}

\begin{IEEEkeywords}
component, formatting, style, styling, insert
\end{IEEEkeywords}

\section{Introduction}
Autonomous vehicles (also known as self-driving vehicles and driverless vehicles) have been researched by many universities, research institutes, internet companies, vehicle companies and companies of other industries around the world since the middle 1980s~\cite{self_driving}. In the last two decades, several crucial of autonomous vehicle research platforms are as follows: the Navlab's mobile platform~\cite{Thorpe199144}, University of Pavia's and Parma's car, ARGO~\cite{Broggi199955}, and UBM's vehicles, VaMoRs and VaMP~\cite{Gregor200248}.

In the last decade, the Defence Advanced Research Projects Agency (DARPA) organized three competitions to accelerate the development of correlation technique about autonomous vehicles~\cite{Brian2016}. In 2004, the first competition named as DARPA Grand Challenge was held in the Mojave Desert, USA~\cite{self_driving}. The goal is to get autonomous vehicles to travel 140 miles of off-road routes as fast as possible~\cite{Brian2016}. Unfortunately, there is no autonomous vehicle which was able to complete the whole journey~\cite{self_driving}.

In 2005, the DARPA Grand Challenge was held again and required autonomous vehicles to travel 132 miles of difficult desert roads across Nevada which contain a mixture of featureless terrain, dust, global positioning system drop-outs, sharp turns, narrow openings, bridges, railroad overpasses, long tunnels, obstacles and a narrow winding mountains road with a 200-foot drop-off~\cite{Buehler2007}. This competition had 23 finalists and 4 cars finished the course within 10-hour limit. Remarkably, the Stanford University's vehicles, Stanley, won the first-place prize, and the Carnegie Mellon University's cars, Sandstorm and H1ghlander, came in second and third place, respectively~\cite{Buehler2007}.

The third competition which is named as the DARPA Urban Challenge was held at the now-closed George Air Force Base, California, USA, in 2007. The objective was for a autonomous vehicles to complete the 60 mile course in less than 6 hours. Additionally, the autonomous vehicles were asked to obey all traffic regulations which avoiding other vehicles including driverless and humandriven vehicles~\cite{buehler2009darpa}. There are 11 finalists in this competition and 6 vehicles completed the route within the allotted time limit. Specifically, Boss, the vehicle of Carnegie Mellon University, won the first-place prize, the Stanford University's car, Junior , claimed second place, and the Virginia Tech's car, Odin, finished in third~\cite{buehler2009darpa}. Although the challenges presented by these competitions cannot cover all the challenges encountered in everyday traffic, they have been hailed as milestones in the development of autonomous vehicles~\cite{Brian2016}.

After the DARPA Challenge, there was a flood of driverless events. Relevant examples include: Intelligent Vehicle Future Challenges~\cite{xin2014china}, from 2009 to 2013; Hyundai Autonomous Challenge~\cite{Cerri2011}, which was held in 2010; VisLab Intercontinental Autonomous Challenge~\cite{broggi2012vislab}, in 2010; the Grand Cooperative Driving Challenge (GCDC)~\cite{Englund2016}, in 2011 and 2016 and Public Road Urban Driverless-Car Test~\cite{Broggi2015}, which was held in 2013. At the same time, many industrial and academic teams have invested a lot of research and development energy in the field of autonomous driving. Among them, the industry is represented by the OEMs of Ford, Toyota, Hyundai and other car companies, as well as IT and emerging companies such as Waymo, Tesla, Uber, Intel, Baidu, Pony.ai, etc., and have developed various autonomous vehicle platforms based on commercialization goals; Academia, including CMU, Stanford, UC Berkeley, Tsinghua University, Tongji University, Southern University of Science and Technology and other major domestic and foreign universities have carried out a series of researches around the key technical fields of autonomous driving.

In order to regulate the application of unmanned driving technology, the National Highway Transportation Safety Administration of the United States Department of Transportation has divided the level of automatic driving (based on the SAE International Standard J3016~\cite{sae2018taxonomy}):
\begin{enumerate}
	\item Level 0 (manual driving): completely controlled by a human driver;
	\item Level 1 (assisted driving): The driving environment provides support for one of the steering wheel and acceleration and deceleration operations, and the rest is operated by humans. Contains basic auxiliary driving, such as adaptive cruise control (ACC), anti-lock brake system (ABS), electronic stability control system (ESC);
	\item Level 2 (semi-automatic driving): The driving environment provides support for multiple operations in the steering wheel and acceleration and deceleration, and the rest is operated by humans.  Contains some advanced auxiliary driving functions, such as a horizontal/vertical control system with minimal risk, emergency braking, etc.;
	\item Level 3 (Highly Autonomous Driving): The unmanned driving system completes all operations, but requires the human driver to take over when leaving the operational scene of unmanned driving.
	\item Level 4 (Ultra-high auto-driving): An unmanned driving system with limited roads and environmental conditions, and does not require human drivers to respond to system requests.
	\item Level 5 (Fully Automated Driving): Unmanned driving system that does not limit roads and environmental conditions
\end{enumerate}

The perception system and the decision-making system are two main parts of the autonomy system architecture of autonomous vehicle~\cite{Brian2016}. To limit the scope of this survey, We focus on some aspects of decision-making system, which includes route planning, path planning, behavior selection, motion planning, obstacle avoidance and control, in particular, for systems falling into the automation level of 3 and above~\cite{self_driving}.

The remainder of the paper id structured as follows:
\begin{enumerate}
	\item Overview of the Decision-Making Hierarchy
	\item Route planning
	\item Path planning
	\item Behavior selector
	\item Motion planning
	\item Obstacle Avoidance and control
	\item Conclusion
\end{enumerate}


\section{Background}
\begin{enumerate}
	\item Identify the problems that you are going to address/review
	\item State the motivation that why do this review
\end{enumerate}
\section{Review different methods or categories}
\begin{enumerate}
	\item Categorize the existing work
	\item The description of each methods, advantage and disadvantage of each method
\end{enumerate}
\section{Discussion}
\begin{enumerate}
	\item Summary of progress
	\item Problem unsolved 
	\item Research directions
\end{enumerate}


\section*{Acknowledgment}

The preferred spelling of the word ``acknowledgment'' in America is without 
an ``e'' after the ``g''. Avoid the stilted expression ``one of us (R. B. 
G.) thanks $\ldots$''. Instead, try ``R. B. G. thanks$\ldots$''. Put sponsor 
acknowledgments in the unnumbered footnote on the first page.


\bibliographystyle{IEEEtran}
\bibliography{decision_making} 

\end{document}

